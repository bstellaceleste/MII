%\title{University of Bristol Thesis Template}
\RequirePackage[l2tabu]{nag}		% Warns for incorrect (obsolete) LaTeX usage
%

\documentclass[a4paper,12pt,leqno,oneside]{memoir} %add 'draft' to turn draft option on (see below)
\usepackage[T1]{fontenc}
%
%
% Adding metadata:
\usepackage[usenames,dvipsnames,svgnames,table]{xcolor}
\usepackage{calligra}
\usepackage{tikz}
\usepackage{adjustbox}
%\usepackage{hyperref}
\usetikzlibrary{arrows,shapes,snakes,
		       automata,backgrounds,
		       petri,topaths}%geometric/algebraic description.
%\usepackage[version=0.96]{pgf}			%PGF/TikZ is a tandem of languages for producing vector graphics from a 

\usepackage{tcolorbox}
\usepackage{enumitem}
\usepackage{pifont}
\tcbuselibrary{skins}
\usepackage{psvectorian}
\renewcommand*{\psvectorianDefaultColor}{Blue}%
\usepackage{datetime}

% When draft option is on. 
\ifdraftdoc 
	\usepackage{draftwatermark}				%Sets watermarks up.
	\SetWatermarkScale{0.3}
	\SetWatermarkText{\bf Draft: \today}
\fi
%
% Declare figure/table as a subfloat.
\newsubfloat{figure}
\newsubfloat{table}
\usepackage{booktabs}
\usepackage{siunitx}

\usetikzlibrary{chains}
\usepackage{multirow}
\usepackage{algorithm}
\usepackage[noend]{algpseudocode}
\makeatletter
\def\BState{\State\hskip-\ALG@thistlm}
\makeatother

% Better page layout for A4 paper, see memoir manual.
\settrimmedsize{297mm}{210mm}{*}
\setlength{\trimtop}{0pt} 
\setlength{\trimedge}{\stockwidth} 
\addtolength{\trimedge}{-\paperwidth} 
\settypeblocksize{634pt}{448.13pt}{*} 
\setulmargins{4cm}{*}{*} 
\setlrmargins{*}{*}{1.5} 
\setmarginnotes{17pt}{51pt}{\onelineskip} 
\setheadfoot{\onelineskip}{2\onelineskip} 
\setheaderspaces{*}{2\onelineskip}{*} 
\checkandfixthelayout
%
\frenchspacing
% Font with math support: New Century Schoolbook
\usepackage{fouriernc}
\usepackage[french]{minitoc}
\setcounter{secnumdepth}{4}
\setcounter{tocdepth}{4}
\setcounter{minitocdepth}{4}
\usepackage{fancybox}
\usepackage{makeidx}
%%%%%%%%%%%%%%%%%%%%%%%%%%%%%%%%%%%%%%%%%%%%%%%%%%%%%%%%%%%%%%%%%%%%%%%%%%%%%%%%%%%%%%%%%%%%%%%%%%%%%%%%%%%%%%%%%
\definecolor{marron}{RGB}{60,30,10}
\definecolor{darkblue}{RGB}{0,0,80}
\definecolor{lightblue}{RGB}{80,80,80}
\definecolor{darkgreen}{RGB}{0,80,0}
\definecolor{darkgray}{RGB}{0,80,0}
\definecolor{darkorange}{RGB}{165,42,42}
\definecolor{darkred}{RGB}{80,0,0}
\definecolor{shadecolor}{rgb}{0.97,0.97,0.97}
\usepackage{fourier-orns}
\usepackage[T1]{fontenc}
%%%%%%%%%%%%%%%%%%%%%%%%%%%%%%%%%%%%%%%%%%%%%%%%%%%%%%%%%%%%%%%%%%%%%%%%%%%%%%%%%%%%%%%%%%%%%%%%%%%%%%%%%%%%%%%%%%%%%
\makeatletter
\def\headrule{{\color{darkorange}\raisebox{-2.1pt}[10pt][10pt]{\leafright} \hrulefill \raisebox{-2.1pt}[10pt][10pt]{~~~\decofourleft \decotwo\decofourright~~~} \hrulefill \raisebox{-2.1pt}[10pt][10pt]{ \leafleft}}}
\makeatother

\usepackage[
final,
stretch=10,
protrusion=true,
tracking=true,
spacing=on,
kerning=on,
expansion=true]{microtype}

\newcommand{\ornamento}{\vspace{-2em} \noindent \textcolor{darkorange}{\hrulefill \raisebox{-2.5pt}[10pt][10pt]{\leafright \decofourleft \decothreeleft  \aldineright \decotwo \floweroneleft \decoone   \floweroneright \decotwo \aldineleft\decothreeright \decofourright \leafleft} \hrulefill}}
\newcommand{\ornpar}{\noindent \textcolor{darkorange}{\hrulefill \raisebox{-1.9pt}[10pt][10pt]{}}}
\newcommand{\ornimpar}{\textcolor{darkorange}{\raisebox{-1.9pt}[10pt][10pt]{} \hrulefill \raisebox{-1.9pt}[10pt][10pt]{}}}

\OnehalfSpacing 
%
% Sets numbering division level
\setsecnumdepth{subsection} 
\maxsecnumdepth{subsubsection}
%
% Chapter style (taken and slightly modified from Lars Madsen Memoir Chapter 
% Styles document
\usepackage{calc,soul,fourier}
\makeatletter 
\newlength\dlf@normtxtw 
\setlength\dlf@normtxtw{\textwidth} 
\newsavebox{\feline@chapter} 
\newcommand\feline@chapter@marker[1][4cm]{%
	\sbox\feline@chapter{% 
		\resizebox{!}{#1}{\fboxsep=1pt%
			\colorbox{darkorange}{\color{white}\thechapter}% 
		}}%
		\rotatebox{90}{% 
			\resizebox{%
				\heightof{\usebox{\feline@chapter}}+\depthof{\usebox{\feline@chapter}}}% 
			{!}{\scshape\so\@chapapp}}\quad%
		\raisebox{\depthof{\usebox{\feline@chapter}}}{\usebox{\feline@chapter}}%
} 
\newcommand\feline@chm[1][4cm]{%
	\sbox\feline@chapter{\feline@chapter@marker[#1]}% 
	\makebox[0pt][c]{% aka \rlap
		\makebox[1cm][r]{\usebox\feline@chapter}%
	}}
\makechapterstyle{daleifmodif}{
	\renewcommand\chapnamefont{\normalfont\LARGE\scshape\raggedleft\so} 
	\renewcommand\chaptitlefont{\normalfont\LARGE\bfseries\scshape} 
	\renewcommand\chapternamenum{} \renewcommand\printchaptername{} 
	\renewcommand\printchapternum{\null\hfill\feline@chm[2.5cm]\par} 
	\renewcommand\afterchapternum{\par\vskip\midchapskip} 
	\renewcommand\printchaptertitle[1]{\color{darkorange}\chaptitlefont\raggedleft ##1\par}
} 
\makeatother 

%\renewcommand\refname{\textbf{Reference}}
\chapterstyle{daleifmodif}
%
% UoB guidelines:
%
% The pages should be numbered consecutively at the bottom centre of the
% page.
\renewcommand{\sectionmark}[1]{\markright{\thesection~- ~#1}}
\renewcommand{\chaptermark}[1]{\markboth{\chaptername~\thechapter~-~ #1}{}}

\makepagestyle{myvf} 
\makeoddfoot{myvf}{}{\ornimpar \\ \large \hfill \sffamily\bf \textcolor{black}{\thepage}
}{} 
\makeevenfoot{myvf}{}{\ornpar   \\ \large  \sffamily\bf \textcolor{black}{\thepage}  \hfill}{} 
%\makeheadrule{myvf}{\textwidth}{\normalrulethickness} 
\makeevenhead{myvf}{\leftmark}{\ornamento}{\rightmark} 
\makeoddhead{myvf}{\leftmark}{\ornamento}{\rightmark}
\pagestyle{myvf}

\newcommand{\clearemptydoublepage}{\newpage{\thispagestyle{empty}\cleardoublepage}}
%
% Creates indexes for Table of Contents, List of Figures, List of Tables and Index
\makeindex
%%%%%%%%%%%%%%%%%%%%%%%%%%%%%%%%%%%%%%%%%%%%%%%%%%%%%%%%%%%%%%%%%%%%%%%%%%%%%%%%%%%%%%%%%%%%%%%%%%%%%%
\usepackage{import}
\usepackage{lipsum}					%Needed to create dummy text
\usepackage{subcaption}
\usepackage{amsfonts} 					%Calls Amer. Math. Soc. (AMS) fonts
\usepackage[centertags]{amsmath}			%Writes maths centred down
\usepackage{stmaryrd}					%New AMS symbols
\usepackage{pgfplots}
\pgfplotsset{yticklabel style={text width=3em,align=right}}

\usepackage{amssymb}					%Calls AMS symbols
\usepackage{amsthm}					%Calls AMS theorem environment
\usepackage{newlfont}					%Helpful package for fonts and symbols
\usepackage{layouts}					%Layout diagrams
\usepackage{graphicx}					%Calls figure environment
\usepackage{longtable,rotating}			%Long tab environments including rotation. 
\usepackage[applemac, utf8]{inputenc}			%Needed to encode non-english characters 
\usepackage{color}
\usepackage{colortbl}					%Makes coloured tables
\usepackage{wasysym}					%More math symbols
\usepackage{mathrsfs}					%Even more math symbols
\usepackage{float}						%Helps to place figures, tables, etc. 
\usepackage{verbatim}					%Permits pre-formated text insertion
\usepackage{upgreek }					%Calls other kind of greek alphabet
\usepackage{latexsym}					%Extra symbols
%\usepackage{natbib}
%\usepackage[maxnames=1]{apalike}
\usepackage{biblatex}
\addbibresource{frontmatter/bibtex.bib}   
\usepackage[spanish,english]{babel}		%For languages characters and hyphenation
\usepackage{memhfixc}					
\usepackage{color}     
\usepackage{enumerate}					%For enumeration counter
\usepackage{footnote}					%For footnotes
\usepackage{afterpage}
\usepackage[object=vectorian]{pgfornament}
\usetikzlibrary{shapes.geometric,calc}
\definecolor{fondpaille}{cmyk}{0,0,0.1,0}
\usepackage{pagecolor}
\usepackage{microtype}					%Makes pdf look better.
\usepackage{rotfloat}					%For rotating and float environments as tables, 
\usepackage{alltt}						%LaTeX commands are not disabled in 
\usepackage[acronym]{glossaries}
\usepackage{acronym}
\usepackage{chngcntr}
\usepackage[section]{placeins}  
\usepackage{titletoc}
%%%%%%%%%%%%%%%%%%%%%%%%%%%%%%%%%%%%%%%%%%%%%%%%%%%%%%%%%%%%%%%%%%%%%%%%%%%%%%%%%%%%%%%%%%%%%%%
\tcbset{
    Baystyle/.style={
        sharp corners,
        enhanced,
        boxrule=6pt,
        colframe=OliveGreen,
        height=600pt,
        width=500pt,
        borderline={8pt}{-11pt}{},
    }
}


%							
%Reduce widows  (the last line of a paragraph at the start of a page) and orphans 
% (the first line of paragraph at the end of a page)
\widowpenalty=1000
\clubpenalty=1000
%
% New command definitions for my thesis
%
\newcommand{\keywords}[1]{\par\noindent{\small{\bf Keywords:} #1}} %Defines keywords small section
\newcommand{\parcial}[2]{\frac{\partial#1}{\partial#2}}                             %Defines a partial operator
\newcommand{\vectorr}[1]{\mathbf{#1}}                                                        %Defines a bold vector
\newcommand{\vecol}[2]{\left(                                                                         %Defines a column vector
	\begin{array}{c} 
		\displaystyle#1 \\
		\displaystyle#2
	\end{array}\right)}
\newcommand{\mados}[4]{\left(                                                                       %Defines a 2x2 matrix
	\begin{array}{cc}
		\displaystyle#1 &\displaystyle #2 \\
		\displaystyle#3 & \displaystyle#4
	\end{array}\right)}
\newcommand{\pgftextcircled}[1]{                                                                    %Defines encircled text
    \setbox0=\hbox{#1}%
    \dimen0\wd0%
    \divide\dimen0 by 2%
    \begin{tikzpicture}[baseline=(a.base)]%
        \useasboundingbox (-\the\dimen0,0pt) rectangle (\the\dimen0,1pt);
        \node[circle,draw,outer sep=0pt,inner sep=0.1ex] (a) {#1};
    \end{tikzpicture}
}
\newcommand{\ra}[1]{\renewcommand{\arraystretch}{#1}}
\newcommand{\myrowcolour}{\rowcolor[gray]{0.925}}

\newcommand{\range}[1]{\textnormal{range }#1}                                             %Defines range operator
\newcommand{\innerp}[2]{\left\langle#1,#2\right\rangle}                                 %Defines inner product
\newcommand{\prom}[1]{\left\langle#1\right\rangle}                                         %Defines average operator
\newcommand{\tra}[1]{\textnormal{tra} \: #1}                                                       %Defines trace operator
\newcommand{\sign}[1]{\textnormal{sign\,}#1}                                                   %Defines sign operator
\newcommand{\sech}[1]{\textnormal{sech} #1}                                                  %Defines sech
\newcommand{\diag}[1]{\textnormal{diag} #1}                                                    %Defines diag operator
\newcommand{\arcsech}[1]{\textnormal{arcsech} #1}                                       %Defines arcsech
\newcommand{\arctanh}[1]{\textnormal{arctanh} #1}                                         %Defines arctanh
%Change tombstone symbol

\newcommand{\blackged}{\hfill$\blacksquare$}
\newcommand{\whiteged}{\hfill$\square$}
\newcounter{proofcount}
\renewenvironment{proof}[1][\proofname.]{\par
 \ifnum \theproofcount>0 \pushQED{\whiteged} \else \pushQED{\blackged} \fi%
 \refstepcounter{proofcount}
 \normalfont 
 \trivlist
 \item[\hskip\labelsep
       \itshape
   {\bf\em #1}]\ignorespaces
}{%
 \addtocounter{proofcount}{-1}
 \popQED\endtrivlist
}
%
%
% New definition of square root:
% it renames \sqrt as \oldsqrt
\let\oldsqrt\sqrt
% it defines the new \sqrt in terms of the old one
\def\sqrt{\mathpalette\DHLhksqrt}
\def\DHLhksqrt#1#2{%
\setbox0=\hbox{$#1\oldsqrt{#2\,}$}\dimen0=\ht0
\advance\dimen0-0.2\ht0
\setbox2=\hbox{\vrule height\ht0 depth -\dimen0}%
{\box0\lower0.4pt\box2}}
%
% My caption style
\newcommand{\mycaption}[2][\@empty]{
	\captionnamefont{\scshape} 
	\changecaptionwidth
	\captionwidth{0.9\linewidth}
	\captiondelim{.\:} 
	\indentcaption{0.75cm}
	\captionstyle[\centering]{}
	\setlength{\belowcaptionskip}{10pt}
	\ifx \@empty#1 \caption{#2}\else \caption[#1]{#2}
}
\newcommand*\circled[1]{\tikz[baseline]{\node[shape=circle,draw,inner sep=2pt] (char) {#1};}}

%
% My subcaption style
\newcommand{\mysubcaption}[2][\@empty]{
	\subcaptionsize{\small}
	\hangsubcaption
	\subcaptionlabelfont{\rmfamily}
	\sidecapstyle{\raggedright}
	\setlength{\belowcaptionskip}{10pt}
	\ifx \@empty#1 \subcaption{#2}\else \subcaption[#1]{#2}
}
%
\newcommand{\sectionlinetwo}[2]{%
  \nointerlineskip \vspace{.5\baselineskip}\hspace{\fill}
  {\color{#1}
    \resizebox{0.5\linewidth}{2ex}
    {{%
    {\begin{tikzpicture}
    \node  (C) at (0,0) {};
    \node (D) at (9,0) {};
    \path (C) to [ornament=#2] (D);
    \end{tikzpicture}}}}}%
    \hspace{\fill}
    \par\nointerlineskip \vspace{.5\baselineskip}
  }
%An initial of the very first character of the content
\usepackage{Carrickc,ArtNouvc,Kramer,Konanur,Rothdn,Starburst}
\usepackage{hyperref}
\usepackage{lettrine}
%\renewcommand\LettrineFontHook{\Rothdnfamily}
\newcommand{\initial}[1]{%
    \setcounter{DefaultLines}{3}%
    \renewcommand{\rmdefault}{fnc}%
    \renewcommand\LettrineFontHook{\Starburstfamily}%
	\lettrine[lines=2,lhang=0.45,nindent=0.5em]{
		\color{darkorange}
     		{\textsc{#1}}}{}}
     		
\newcommand{\corner}[1]{%
  \begin{tikzpicture}[color=darkgray,remember picture, overlay]
    \node[anchor=north west] at (current page.north west){%
      \pgfornament[width=2cm]{#1}};
    \node[anchor=north east] at (current page.north east){%
      \pgfornament[width=2cm,symmetry=v]{#1}};
    \node[anchor=south west] at (current page.south west){%
      \pgfornament[width=2cm,symmetry=h]{#1}};
    \node[anchor=south east] at (current page.south east){%
      \pgfornament[width=2cm,symmetry=c]{#1}};
  \end{tikzpicture}%
}
\newcommand{\cornerplus}[2]{%
  \begin{tikzpicture}[color=darkgray,remember picture, overlay]
    \node[anchor=north west] at (current page.north west){%
      \pgfornament[width=2cm]{#1}};
    \node[anchor=north east] at (current page.north east){%
      \pgfornament[width=2cm,symmetry=v]{#1}};
    \node[anchor=south west] at (current page.south west){%
      \pgfornament[width=2cm,symmetry=h]{#1}};
    \node[anchor=south east] at (current page.south east){%
      \pgfornament[width=2cm,symmetry=c]{#1}};
    \node[anchor=north] at (current page.north){%
      \pgfornament[width=6.5cm,symmetry=h]{#2}};
    \node[anchor=south] at (current page.south){%
      \pgfornament[width=6.5cm]{#2}};
  \end{tikzpicture}%
}
\newcommand{\pt}[1]{%
  \begin{tikzpicture}[color=darkgray,remember picture, overlay]
    \node[anchor=north] at (current page.north){%
      \pgfornament[symmetry=h]{#1}};
    \node[anchor=south] at (current page.south){%
      \pgfornament[symmetry=h]{#1}};
  \end{tikzpicture}%
}
%
% Theorem styles used in my thesis
%
\AtBeginDocument{\renewcommand{\bibname}{Références}}
\AtBeginDocument{\renewcommand{\refname}{Références}}
\AtBeginDocument{\renewcommand{\chaptername}{Chapitre}}
\AtBeginDocument{\renewcommand{\mtctitle}{Sommaire}}
%%%%%%%%%%%%%%%%%%%%%%%%%%%%%%%%%%%%%%%%%%%%%%%%%%%%%%%%%%%%%%%%%%%%%%%%%%%%%%%%%%%%%%%%%%%%%%%%%%%%%%%%%%
\theoremstyle{plain}
\newtheorem{theo}{Theorem}[chapter]
\theoremstyle{plain}
\newtheorem{prop}{Proposition}[chapter]
\theoremstyle{plain}
\theoremstyle{definition}
\newtheorem{dfn}{Definition}[chapter]
\theoremstyle{plain}
\newtheorem{lema}{Lemma}[chapter]
\theoremstyle{plain}
\newtheorem{cor}{Corollary}[chapter]
\theoremstyle{plain}
\newtheorem{resu}{Result}[chapter]
% Hyphenation for some words
%
\hyphenation{vir-tua-li-sées}
%
%%%%%%%%%%%%%%%%%%%%%%%%%%%%%%%%%%%%%%%%%%%%%%%%%%%%%%%%%%%%%%%%%%%%%%%%%%%%%%%%%%%%%%%%%%%%%%%%%%%%%%%%%%%
\addto\captionsenglish{%
   \renewcommand\listfigurename{Liste des figures}}   
\addto\captionsenglish{%
   \renewcommand\listtablename{List des tableaux}}
 
\renewcommand{\baselinestretch}{1.5}
%%%%%%%%%%%%%%%%%%%%%%%%%%%%%%%%%%%%%%%%%%%%%%%%%%%%%%%%%%%%%%%%%%%%%%%%%%%%%%%%%%%%%%%%%%%%%%
\begin{document}
\sloppy

\frontmatter
\pagenumbering{roman}
%
%
% File: Title.tex
% Author: V?ctor Bre?a-Medina
% Description: Contains the title page
%
% UoB guidelines:
% 
% At the top of the title page, within the margins, the dissertation should give the title and, if 
% necessary, sub-title and volume number. If the dissertation is in a language other than English, the 
% title must be given in that language and in English. The full name of the author should be in the 
% centre of the page. At the bottom centre should be the words ?A dissertation submitted to the 
% University of Bristol in accordance with the requirements for award of the degree of ? in the 
% Faculty of ...?, with the name of the school and month and year of submission. The word count of 
% the dissertation (text only) should be entered at the bottom right-hand side of the page.
%
%
%\hspace{5cm}
%\pagecolor{fondpaille}%\afterpage{\nopagecolor}
\begin{titlingpage}
\begin{SingleSpace}

%\calccentering{\unitlength} 
\begin{adjustwidth*}{\unitlength}{-\unitlength}
\vspace*{13mm}
\vspace{-4cm}

\begin{center}
\begin{adjustbox}{width={\textwidth},totalheight={\textheight},keepaspectratio}%
 	\begin{tabular}{c c c}
				{UNIVERSITE DE YAOUNDE I} & \multirow{7}{*}{\includegraphics[ width =3cm , height =3cm]{logos/uy1.png}} & {UNIVERSITY OF YAOUNDE I}\\
				%& & \\
				\textbf{*******} &    & \textbf{*******} \\
				%& & \\
				ECOLE NATIONALE SUPERIEURE & & NATIONAL ADVANCED SCHOOL\\ 
			    POLYTECHNIQUE	 &    &  OF ENGINEERING \\
				%& & \\
				\textbf{*******} &    & \textbf{*******} \\
				%& & \\
				DEPARTEMENT DE GENIE& & DEPARTMENT OF COMPUTER\\ 
				 INFORMATIQUE & &  ENGINEERING \\
	\end{tabular}
\end{adjustbox}
\end{center}

%\hspace{3cm}
\vspace{0.3cm}
\begin{center}
%\rput(0,0){\pgfornament[scale=.9, color=darkorange]{89}}\\
%%\vspace{0.65cm}
\textcolor{darkorange}{\rule[0.5ex]{\linewidth}{2pt}\vspace*{-\baselineskip}\vspace*{2.9pt}}
\textcolor{darkorange}{\rule[0.5ex]{\linewidth}{1pt}}\\[\baselineskip]
{\LARGE {\textcolor{darkorange}{\textsc{ technique d'estimation du working set basee sur le page modification logging (PML) }}
} }\\[2mm]
\textcolor{darkorange}{\rule[0.5ex]{\linewidth}{1pt}\vspace*{-\baselineskip}\vspace{3.2pt}}
\textcolor{darkorange}{\rule[0.5ex]{\linewidth}{2pt}}\\[3mm]
%%\rput(0,0){\pgfornament[scale=.9, color=darkorange]{89}}\\[10mm]
{\Large \textbf{End of course dissertation/Master of Engineering}}\\
\vspace{4mm}
{\Large Presented and defended by } \\
\vspace{4mm}
{\large \textsc{\textbf{\textcolor{darkorange}{NOM Prenom}}}}\\
\vspace{6mm}
{\Large In partial fulfilment of the requirements for the award of a:} \\
\vspace{4mm}
{\large \textbf{\textcolor{darkorange}{Master of Engineering in Computer Science}}}\\
\vspace{4mm}
{\Large Under the supervision of:}\\
\vspace{4mm}
{\normalsize \textsc{\textbf{\textcolor{darkorange}{XXXX xxxx, Professor, National Polytechnic Institute of Toulouse}}}}\\
\vspace{4.5mm}
{\normalsize \textsc{\textbf{\textcolor{darkorange}{XXXX xxxx, Associate Professor, National Polytechnic Institute of Toulouse}}}}\\
\vspace{4.5mm}
{\normalsize \textsc{\textbf{\textcolor{darkorange}{XXXX xxxx, Engineer, \acrlong{irit}}}}} \\
\vspace{4.5mm}
{\Large In front of the jury composed of:} \\
\vspace{4.5mm}
\begin{tabular}{>{\centering\arraybackslash}p{16cm}}
{\Large President:} \textbf{{\large \textsc{XXXXX XXXXX, Associate Professor, University of Yaounde I}}} \\ \\

{\Large Examiner:} {\large \textsc{\textbf{Bernab\'{e} BATCHAKUI, Senior Lecturer, University of Yaounde I}}}
\end{tabular}\\
\vspace{6.5mm}
\begin{tabular}{c}
{\Large \textcolor{darkorange}{Academic year 2016-2017}}\\
{\Large \textcolor{darkorange}{Defended the 08th September 2017}}
\end{tabular}

%Front page corner design...read ornaments documentation for more info
%Peterson Yuhala
\begin{tikzpicture}[remember picture, overlay,color=darkorange]
\node[anchor=north west] at (current page.north west){%
\pgfornament[width=2cm]{41}};
\node[anchor=north east] at (current page.north east){%
\pgfornament[width=2cm,symmetry=v]{41}};
\node[anchor=south west] at (current page.south west){%
\pgfornament[width=2cm,symmetry=h]{41}};
\node[anchor=south east] at (current page.south east){%
\pgfornament[width=2cm,symmetry=c]{41}};
\end{tikzpicture}
% \begin{tikzpicture}[remember picture, overlay]
%  \begin{scope}[shift={(current page.south west)},shift={(1,1)},scale=1]
%  \shade[ball color=darkorange,opacity=.6] (0,0) circle (10ex);
%  \shade[ball color=darkorange,opacity=.8] (1.7,1) circle (6ex);
%  \shade[ball color=darkorange,opacity=.8] (1.5,3) circle (2ex);
%  \shade[ball color=darkorange,opacity=.5] (-0.5,3) circle (1ex);
%  \shade[ball color=darkorange,opacity=.8] (1,4) circle (1ex);
%  \shade[ball color=darkorange,opacity=.6] (3.5,2.5) circle (2ex);
%  \shade[ball color=darkorange,opacity=.8] (2.5,3.5) circle (2ex);
%  \end{scope}
%  \end{tikzpicture}
%%%%%%%%%%%%%%%%%%%%%%%%%%%%%%%%%%%%%%%%%%%%%%%%%%%%%%%%%%%%

%\cornerplus{41}{88}
\end{center}


\end{adjustwidth*}

\end{SingleSpace}
\end{titlingpage}

\cleardoublepage

%
% File: Title.tex
% Author: V?ctor Bre?a-Medina
% Description: Contains the title page
%
% UoB guidelines:
% 
% At the top of the title page, within the margins, the dissertation should give the title and, if 
% necessary, sub-title and volume number. If the dissertation is in a language other than English, the 
% title must be given in that language and in English. The full name of the author should be in the 
% centre of the page. At the bottom centre should be the words ?A dissertation submitted to the 
% University of Bristol in accordance with the requirements for award of the degree of ? in the 
% Faculty of ...?, with the name of the school and month and year of submission. The word count of 
% the dissertation (text only) should be entered at the bottom right-hand side of the page.
%
%
%\hspace{5cm}
%\pagecolor{fondpaille}\afterpage{\nopagecolor}
\begin{titlingpage}
\begin{SingleSpace}

%\calccentering{\unitlength} 
\begin{adjustwidth*}{\unitlength}{-\unitlength}
\begin{figure*}
\centering 
\includegraphics[scale=3.7]{logos/irit}
\end{figure*}
\vspace{0.5cm}
\begin{center}
\rput(0,0){\pgfornament[scale=.9, color=darkorange]{89}}\\
\vspace{0.65cm}
{\LARGE{\textcolor{darkorange} {\textsc{Technique d'estimation du working set basee sur le Page Modification Logging (PML)}
}}}\\[2mm]
\rput(0,0){\pgfornament[scale=.9, color=darkorange]{89}}\\
\vspace{6mm}
{\Large \textbf{End of course dissertation/Master of Engineering}}\\
\vspace{6mm}
{\Large Presented and defended by } \\
\vspace{6mm}
{\Large \textsc{\textbf{\textcolor{darkorange}{XXXX XXXXX }}}}\\
\vspace{8mm}
{\Large In partial fulfilment of the requirements for the award of a:} \\
\vspace{6mm}
{\Large \textbf{\textcolor{darkorange}{Master of Engineering in Computer Science }}}\\
\vspace{6mm}
\vspace{4.5mm}
\begin{tabular}{c}
{\Large \textcolor{darkorange}{Academic year 2017-2018}}\\
{\Large \textcolor{darkorange}{Defended the 08 July 2018}}
\end{tabular}\\
\vspace{.5cm}
{\includegraphics[scale=.23]{logos/labri}}

\end{center}

\end{adjustwidth*}

\end{SingleSpace}
\end{titlingpage}

\cleardoublepage
%
% file: dedication.tex
% author: V?ctor Bre?a-Medina
% description: Contains the text for thesis dedication
%

\chapter*{Dédicace}
%\begin{SingleSpace}
\addstarredchapter{Dédicace}
%\adjustmtc
\vspace{5cm}
\begin{center}  
\rput[r](-3pt,3pt){\pgfornament[scale=.65, color=darkorange]{72}}
\large{\textcolor{darkorange}{\textbf{A ma famille}}}%
\rput[l](3pt,3pt){\pgfornament[scale=.65, color=darkorange]{73}}\\
\vspace{5mm}
\rput(0,0){\pgfornament[scale=.8, color=darkorange]{85}}

\end{center}
%\end{SingleSpace}
\clearpage

\clearemptydoublepage
%
%
% File: declaration.tex
% Author: V?ctor Bre?a-Medina
% Description: Contains the declaration page
%
% UoB guidelines:
%
% Author's declaration
%
% I declare that the work in this dissertation was carried out in accordance
% with the requirements of the University's Regulations and Code of Practice
% for Research Degree Programmes and that it has not been submitted for any
% other academic award. Except where indicated by specific reference in the
% text, the work is the candidate's own work. Work done in collaboration with,
% or with the assistance of, others, is indicated as such. Any views expressed
% in the dissertation are those of the author.
%
% SIGNED: .............................................................
% DATE:..........................
%
\chapter*{Remerciements}
\addstarredchapter{Remerciements}
%\adjustmtc
%\markboth{Remerciements}{}
\begin{SingleSpace}
%\begin{quote}
Ma gratitude va à l'endroit de tous ceux qui ont contribué à mon éducation, à ma formation et à la réalisation de ce travail :

\begin{itemize}
	\item \textbf{Pr Claude TANGA}, pour l’honneur qu’il me fait en acceptant de présider ce jury.
    \item \textbf{Pr Thomas BOUETOU BOUETOU}, pour son encadrement, son dévouement dans l'enseignement et la direction de notre département de génie informatique.
    \item \textbf{Pr Alain TCHANA} et \textbf{Pr Laurent REVEILLERE}, pour leur disponibilité, leurs directives, leurs conseils, leur suivi, ainsi que pour l’opportunité qu’ils m’ont offerte d’effectuer ce stage.
    \item \textbf{Pr Thomas DJOTIO NDIE}, pour l’honneur qu’il me fait en acceptant d’examiner scientifiquement ce travail. 
    \item La Direction de l’Institut de Recherche en Informatique de Toulouse, pour m’avoir permis d’effectuer mon stage dans le sein de son illustre institution.
    \item Le corps enseignant de l’École Nationale Supérieure Polytechnique de Yaoundé, en particulier celui du Département de Génie Informatique, pour la formation et le savoir qu’il s’est dévoué à me transmettre. Je remercie tout spécialement \textbf{Pr Thomas BOUETOU BOUETOU, Pr Thomas DJOTIO NDIE, Dr Bernabé BATCHAKUI et Dr Georges Edouard KOUAMOU}.
    \item Les membres de l’équipe SEPIA de l’IRIT, équipe dans laquelle j’ai effectué ce \break stage : \textbf{Boris TEABE, Djob MVONDO, Lavoisier WAPET, Kevin JIOKENG, Vlad NITU, Grégoire TODESCHI, Mathieu BACOU} pour leur collaboration et l’aide qu’ils m’ont apportée pendant le déroulement de ce stage.
    \item Mon camarade \textbf{Peterson YUHALA}, avec qui nous avons vécu cette belle expérience de stagiaires.
\end{itemize}

\noindent Je ne saurais manquer d'adresser ma reconnaissance envers : 
\begin{itemize}
	\item \textbf{Mes parents, M. Désiré BITCHEBE} et \textbf{Mme Cécile FETNGO}, pour leur soutien, leurs conseils et leur perpétuelle présence pour moi.
	\item \textbf{Mes soeurs}, pour leur présence, leurs encouragements et leur amour.
    \item \textbf{Stéphane TJOMB}, pour son accompagnement chaque jour, son soutien inébranlable, son écoute et son affection.
    \item La promotion \textbf{GI2018}. Ensemble nous avons partagé des connaissances et des idées. Ensemble nous avons partagé de précieux moments. 
    \item Tous mes amis, qui m’encouragent et me soutiennent chaque jour.
\end{itemize}
%\end{quote}
\end{SingleSpace}
\clearpage

\clearemptydoublepage
%
\chapter*{Liste des sigles et acronymes}
\addstarredchapter{Liste des sigles et acronymes}
\markboth{Liste des sigles et acronymes}{Liste des sigles et acronymes}

% Remember to use italic fonts for foreign words
\begin{acronym}[ENSEEIHT] % Specify the longest acronym in order to set the first column width

\acro{A/D}{Accessed and Dirty}
\acro{CPU}{Core Processing Unit}
\acro{DCs}{Datacenters}
\acro{ENSEEIHT}{Ecole Nationale Supérieure d'Electrotechnique, d'Electronique, d'Informatique, d'Hydraulique et des Télécommunications}
\acro{ENSPY}{Ecole Nationale Supérieure Polytechnique de Yaoundé}
\acro{EPT}{Extended Page Table}
\acro{gla}{guest linear address}
\acro{gPA}{guest Physical Address}
\acro{gPT}{guest Page Table}
\acro{gVA}{guest Virtual Address}
\acro{hPA}{host Physical Address}
\acro{hPT}{host Page Table}
\acro{HVM}{Hardware Virtual Machine}
\acro{INPT}{Institut National Polytechnique de Toulouse}
\acro{Intel VT}{Intel Virtual Technology}
\acro{IRIT}{Institut de Recherche en Informatique de Toulouse}
\acro{MMU}{Memory Management Unit}
\acro{NPT}{Nested Page Table}
\acro{OS}{Operating System}
\acro{PML}{Page Modification Logging}
\acro{SLA}{Service Level Agreement}
\acro{SLAT}{Second Level Address Translation}
\acro{TLB}{Translation Lookaside Buffer}
\acro{Vaddr}{Virtual address}
\acro{vCPU}{virtual Core Processing Unit}
\acro{VMCS}{Virtual Machine Control Strucutre}
\acro{VMM}{Virtual Machine Monitor}
\acro{VMs}{Machines Virtuelles}
\acro{VT-x}{Intel Virtual Technology Extensions}
\acro{WSS}{Working Set Size}

\end{acronym}


% To cite an acronym in the text : \ac{ASK}
% To cite an acronym in the text without footnote : \acs{ASK}

%%% Local Variables: 
%%% mode: latex
%%% TeX-master: "../phdthesis"
%%% End: 

\clearemptydoublepage
%
%
% File: abstract.tex
% Author: V?ctor Bre?a-Medina
% Description: Contains the text for thesis abstract
%
% UoB guidelines:
%
% Each copy must include an abstract or summary of the dissertation in not
% more than 300 words, on one side of A4, which should be single-spaced in a
% font size in the range 10 to 12. If the dissertation is in a language other
% than English, an abstract in that language and an abstract in English must
% be included.

\chapter*{Résumé}
\addstarredchapter{Résumé}
%\adjustmtc
%\begin{vcenterpage}
{
%\setstretch{}
\noindent\rule[3pt]{\textwidth}{1pt}
\vspace{.2cm}

\initial{C}es dernières années ont vu la \textbf{\emph{virtualisation}} s'imposer comme technologie de base dans les datacenters. Dans ces derniers, les utilisateurs ont tendance à surdimensionner les ressources de leurs applications, ce qui a pour conséquences : un gaspillage de la mémoire, un faible taux de consolidation et une proportionnalité énergétique faible. La mémoire étant la ressource critique, il serait avantageux d’allouer à une \textbf{\emph{machine virtuelle}} la quantité exacte de mémoire dont elle a besoin à un instant donné. Il faudrait donc être capable d’anticiper sa demande (aussi bien à la hausse qu’à la baisse), i.e. de connaître à tout moment la taille de son \textbf{\emph{working set}}. De nombreuses techniques d'estimation du working set existent déjà, basées sur des solutions dites logicielles, qui sont intrusives et/ou actives, et qui surchargent la machine virtuelle et/ou l'hyperviseur. Dans ce travail nous proposons une technique d'estimation qui permet de pallier les problèmes que posent les solutions logicielles. En effet notre approche est basée sur une technologie matérielle : le \textbf{\emph{PML}} (Page Modification Logging). Ceci la laisse totalement externe à la machine virtuelle (non intrusive et non active) et de plus tous les traitements se font au niveau du \textbf{\emph{dom0}} (pas de surcharge de la machine virtuelle ni de l'hyperviseur). 

\paragraph{Mots clés : Virtualisation, Machine virtuelle, Working Set, PML, dom0.}

\vspace{.2cm}
\noindent\rule[3pt]{\textwidth}{1pt}
}
\clearpage
\clearemptydoublepage
%
%
% File: abstract.tex
% Author: V?ctor Bre?a-Medina
% Description: Contains the text for thesis abstract
%
% UoB guidelines:
%
% Each copy must include an abstract or summary of the dissertation in not
% more than 300 words, on one side of A4, which should be single-spaced in a
% font size in the range 10 to 12. If the dissertation is in a language other
% than English, an abstract in that language and an abstract in English must
% be included.

\chapter*{Abstract}
\addstarredchapter{Abstract}
%\adjustmtc
%\begin{vcenterpage}
{
%\setstretch{}
\noindent\rule[3pt]{\textwidth}{1pt}
\vspace{.2cm}

\initial{I}n recent years \textbf{\emph{virtualization}} has emerged as a core technology in the datacenters. In these datacenters, users tend to overdrive the resources of their applications, which leads to a waste of resources, low consolidation and low energy proportionality. Memory being the most critical resource, it would be advantageous to allocate to a virtual machine, the exact amount of memory it needs at a given moment. It is necessary so be able to wait for his request too to know at any time the size of his \textbf{\emph{work set}}. Many techniques of working set size estimation already exist, based on software solutions that are intrusive and / or active, and which overload the virtual machine and / or the hypervisor. In this work we propose a solution which makes it possible to overcome the problems that software solutions pose. Indeed, our approach is based on a feature called page modification logging (\textbf{\emph{PML}}), which leaves it totally external to the virtual machine (non-intrusive and non-active) and all the processes are done at \textbf{\emph{dom0}} level (no overloading of the virtual machine nor of the hypervisor).

\paragraph{Keywords : Virtualization, Working Set, PML, dom0.}

\vspace{.2cm}
\noindent\rule[3pt]{\textwidth}{1pt}
}
\clearpage
\clearemptydoublepage
%


\listoftables*
\addstarredchapter{\listtablename}
\addtocontents{lot}{\par\nobreak\textbf{{\scshape Tableau} \hfill Page}\par\nobreak}
\clearemptydoublepage
%
\listoffigures*
\addstarredchapter{\listfigurename}
\addtocontents{lof}{\par\nobreak\textbf{{\scshape Figure} \hfill Page}\par\nobreak}
\cleardoublepage
%

\renewcommand{\contentsname}{Table des matières}
\maxtocdepth{subsubsection}

%\dominitoc
\tableofcontents*
\markboth{TABLE OF CONTENTS}{}

\clearemptydoublepage
%%%%%%%%%%%%%%%%%%%%%%%%%%%%%%%%%%%%%%%%%%%%%%%%%%%%%%%%%%%%%%%%%%%%%
\mainmatter
%
\clearemptydoublepage
\import{chapters/introduction/}{intro.tex}
\clearemptydoublepage
\import{chapters/1/}{background.tex}
\clearemptydoublepage
\import{chapters/2/}{etat_art.tex}
\clearemptydoublepage
%%%%%%%%%%%%%%%%%%%%%%%%%%%%%%%%%%%%%%%%%%%%%%%%%%%%%%%%%%%%%%%%%%%%%%%
\backmatter
%\bibliographystyle{mla}
%\bibliography{frontmatter/bibtex}
\printbibliography
\clearemptydoublepage
\appendix
\import{chapters/appendices/}{app0A.tex}
\clearemptydoublepage
\clearemptydoublepage
%
\glsaddall
%   
\end{document}