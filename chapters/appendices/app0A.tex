%
% file: localoperator.tex
% author: Victor Brena
% description: Briefly describes properties of the local operator.
%

\chapter{Annex}
\label{app:app01}

\section*{Internship host structure description}
\initial{I}RIT, the Toulouse Research Institute of Computer Science, one of the largest joint research units at the national level, is one of the pillars of research in the \textbf{Midi-Pyr\'{e}n\'{e}es} with its 700 permanent and non-permanent members. Due to its scientific impact and its interactions with other fields, the laboratory is one of the structuring forces of the \acrshort{it} landscape and its applications in the digital world, so regional and national level. The unit is structured into 7 research themes grouping 21 laboratory teams. Its main research axes are:

\begin{enumerate}
    \item  Analysis and synthesis of information.
    \item  Indexing and retrieving information.
    \item  Interaction, Cooperation, Adaptation through Experimentation.
    \item  Reasoning and decision.
    \item  Modeling, algorithms and high-performance computing.
    \item  Architecture, systems, and networks.
    \item  Software development security.
\end{enumerate}


Beyond that, IRIT's influence is reflected in various actions at european and international level, for example, the european laboratory IREP, the french spanish Laboratory for advanced studies in information and perennial cooperation with various countries including the Maghreb, Japan, Armenia, the Unites States of America, etc.

\paragraph{} IRIT present in all Toulouse universities as well as the Institute of Technology of Tarbes Castres provides both coverage of the local territory and all the themes of computer science and its interactions, ranging from infrastructure enabling it to contribute to the structuring of regional research. The laboratory works to establish a continuum from research to valorization, despite the difficulties and limitations inherent in existing devices (industrial property, patent), by imagining and developing innovative forms of collaboration around the concepts of a joint laboratory or a consortium.


